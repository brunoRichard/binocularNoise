% Options for packages loaded elsewhere
% Options for packages loaded elsewhere
\PassOptionsToPackage{unicode}{hyperref}
\PassOptionsToPackage{hyphens}{url}
\PassOptionsToPackage{dvipsnames,svgnames,x11names}{xcolor}
%
\documentclass[
  12pt,
]{article}
\usepackage{xcolor}
\usepackage[left=25mm,right=25mm,top=25mm,bottom=25mm]{geometry}
\usepackage{amsmath,amssymb}
\setcounter{secnumdepth}{-\maxdimen} % remove section numbering
\usepackage{iftex}
\ifPDFTeX
  \usepackage[T1]{fontenc}
  \usepackage[utf8]{inputenc}
  \usepackage{textcomp} % provide euro and other symbols
\else % if luatex or xetex
  \usepackage{unicode-math} % this also loads fontspec
  \defaultfontfeatures{Scale=MatchLowercase}
  \defaultfontfeatures[\rmfamily]{Ligatures=TeX,Scale=1}
\fi
\usepackage{lmodern}
\ifPDFTeX\else
  % xetex/luatex font selection
\fi
% Use upquote if available, for straight quotes in verbatim environments
\IfFileExists{upquote.sty}{\usepackage{upquote}}{}
\IfFileExists{microtype.sty}{% use microtype if available
  \usepackage[]{microtype}
  \UseMicrotypeSet[protrusion]{basicmath} % disable protrusion for tt fonts
}{}
\makeatletter
\@ifundefined{KOMAClassName}{% if non-KOMA class
  \IfFileExists{parskip.sty}{%
    \usepackage{parskip}
  }{% else
    \setlength{\parindent}{0pt}
    \setlength{\parskip}{6pt plus 2pt minus 1pt}}
}{% if KOMA class
  \KOMAoptions{parskip=half}}
\makeatother
% Make \paragraph and \subparagraph free-standing
\makeatletter
\ifx\paragraph\undefined\else
  \let\oldparagraph\paragraph
  \renewcommand{\paragraph}{
    \@ifstar
      \xxxParagraphStar
      \xxxParagraphNoStar
  }
  \newcommand{\xxxParagraphStar}[1]{\oldparagraph*{#1}\mbox{}}
  \newcommand{\xxxParagraphNoStar}[1]{\oldparagraph{#1}\mbox{}}
\fi
\ifx\subparagraph\undefined\else
  \let\oldsubparagraph\subparagraph
  \renewcommand{\subparagraph}{
    \@ifstar
      \xxxSubParagraphStar
      \xxxSubParagraphNoStar
  }
  \newcommand{\xxxSubParagraphStar}[1]{\oldsubparagraph*{#1}\mbox{}}
  \newcommand{\xxxSubParagraphNoStar}[1]{\oldsubparagraph{#1}\mbox{}}
\fi
\makeatother


\usepackage{longtable,booktabs,array}
\usepackage{calc} % for calculating minipage widths
% Correct order of tables after \paragraph or \subparagraph
\usepackage{etoolbox}
\makeatletter
\patchcmd\longtable{\par}{\if@noskipsec\mbox{}\fi\par}{}{}
\makeatother
% Allow footnotes in longtable head/foot
\IfFileExists{footnotehyper.sty}{\usepackage{footnotehyper}}{\usepackage{footnote}}
\makesavenoteenv{longtable}
\usepackage{graphicx}
\makeatletter
\newsavebox\pandoc@box
\newcommand*\pandocbounded[1]{% scales image to fit in text height/width
  \sbox\pandoc@box{#1}%
  \Gscale@div\@tempa{\textheight}{\dimexpr\ht\pandoc@box+\dp\pandoc@box\relax}%
  \Gscale@div\@tempb{\linewidth}{\wd\pandoc@box}%
  \ifdim\@tempb\p@<\@tempa\p@\let\@tempa\@tempb\fi% select the smaller of both
  \ifdim\@tempa\p@<\p@\scalebox{\@tempa}{\usebox\pandoc@box}%
  \else\usebox{\pandoc@box}%
  \fi%
}
% Set default figure placement to htbp
\def\fps@figure{htbp}
\makeatother





\setlength{\emergencystretch}{3em} % prevent overfull lines

\providecommand{\tightlist}{%
  \setlength{\itemsep}{0pt}\setlength{\parskip}{0pt}}



 


\usepackage{indentfirst}
\setlength{\parindent}{15pt}
\setlength{\parskip}{10pt}
\usepackage[font=small]{caption}  % Options: small, footnotesize, scriptsize, etc.
\usepackage[noblocks]{authblk}
\renewcommand*{\Authsep}{, }
\renewcommand*{\Authand}{, }
\renewcommand*{\Authands}{, }
\renewcommand\Affilfont{\small}
% Define a keywords command for inline keywords
\newcommand{\printkeywords}[1]{\noindent\textbf{Keywords:} #1\par}
\makeatletter
\@ifpackageloaded{caption}{}{\usepackage{caption}}
\AtBeginDocument{%
\ifdefined\contentsname
  \renewcommand*\contentsname{Table of contents}
\else
  \newcommand\contentsname{Table of contents}
\fi
\ifdefined\listfigurename
  \renewcommand*\listfigurename{List of Figures}
\else
  \newcommand\listfigurename{List of Figures}
\fi
\ifdefined\listtablename
  \renewcommand*\listtablename{List of Tables}
\else
  \newcommand\listtablename{List of Tables}
\fi
\ifdefined\figurename
  \renewcommand*\figurename{Figure}
\else
  \newcommand\figurename{Figure}
\fi
\ifdefined\tablename
  \renewcommand*\tablename{Table}
\else
  \newcommand\tablename{Table}
\fi
}
\@ifpackageloaded{float}{}{\usepackage{float}}
\floatstyle{ruled}
\@ifundefined{c@chapter}{\newfloat{codelisting}{h}{lop}}{\newfloat{codelisting}{h}{lop}[chapter]}
\floatname{codelisting}{Listing}
\newcommand*\listoflistings{\listof{codelisting}{List of Listings}}
\makeatother
\makeatletter
\makeatother
\makeatletter
\@ifpackageloaded{caption}{}{\usepackage{caption}}
\@ifpackageloaded{subcaption}{}{\usepackage{subcaption}}
\makeatother
\usepackage{bookmark}
\IfFileExists{xurl.sty}{\usepackage{xurl}}{} % add URL line breaks if available
\urlstyle{same}
\hypersetup{
  pdftitle={Neural responses to binocular correlated and anticorrelated noise stimuli},
  pdfauthor={Bruno Richard; Robert F. Hess; Sing Ip Lee; Daniela Marinova; Daniel H. Baker},
  colorlinks=true,
  linkcolor={blue},
  filecolor={Maroon},
  citecolor={Blue},
  urlcolor={Blue},
  pdfcreator={LaTeX via pandoc}}


\title{Neural responses to binocular correlated and anticorrelated noise
stimuli}

  \author[1]{Bruno Richard}
  \author[2]{Robert F. Hess}
  \author[3]{Sing Ip Lee}
  \author[3]{Daniela Marinova}
  \author[3]{Daniel H. Baker}

      \affil[1]{Department of Mathematics and Computer Science, Rutgers
University, Newark, New Jersey, USA}
      \affil[2]{Department of Ophthalmology, McGill University,
Montreal, Qc, Canada}
      \affil[3]{Department of Psychology, University of York, York, UK}
  
\date{}
\begin{document}
\maketitle
\begin{abstract}
Binocular vision fuses monocular inputs into a single percept. To
integrate the input from the eyes, the visual system is thought to
encode the monocular signals with summation and differencing channels.
Here, we investigate the ability of Steady-State Visually Evoked
Potentials (SSVEPs) to record neural responses associated with summation
and differencing mechanisms. We measured responses to binocular noise
stimuli with differing degrees of interocular correlation. Across the
eyes, stimuli could oscillate from (i) no correlation to perfect
interocular correlation (i.e., the same image in both eyes), (ii) no
correlation to perfect interocular correlation with a disparity cue,
(iii) from no correlation to perfect interocular anticorrelation (i.e.,
opposite interocular contrast), and (iv) a control condition with no
interocular correlation. A total of XX observers participated in the
study. Responses to correlated stimuli showed a peak at the fundamental
frequency (3Hz), while adding a disparity cue led to a response at the
fundamental and its second harmonic (6Hz). The amplitude of the
fundamental frequency in the stereocue condition was strongly correlated
with the stereoacuity of observers. Surprisingly, no steady-state
responses were found for interocular anticorrelation; signal-to-noise
ratios at 3Hz were no different from those of the control condition. We
modelled our data using an image-based variant of the two-stage contrast
gain control model of binocular summation. To generate SSVEPs dependent
on interocular correlation, the noise images were filtered with a bank
of log-Gabors that had preferred orientations ranging from 0° to 165°,
in increments of 15°, and preferred spatial frequencies of 0.5, 1, 2, 4,
8, and 16 cycles/°. The monocular filter responses underwent an early
non-linearity and contrast gain control before binocular summation and
binocular difference. The sum and difference responses were fed through
a second non-linear and contrast gain control. The resulting output was
Fourier transformed to generate model SSVEPs.
\end{abstract}


\subsubsection{Median SNRs}\label{median-snrs}

\pandocbounded{\includegraphics[keepaspectratio]{BinocularNoise_VSS_Abstract_files/figure-pdf/unnamed-chunk-1-1.pdf}}

\subsubsection{Condition Comparisons}\label{condition-comparisons}

\pandocbounded{\includegraphics[keepaspectratio]{BinocularNoise_VSS_Abstract_files/figure-pdf/Correlated Noise to Control-1.pdf}}

\subsubsection{``Model'' Responses}\label{model-responses}

\pandocbounded{\includegraphics[keepaspectratio]{BinocularNoise_VSS_Abstract_files/figure-pdf/unnamed-chunk-2-1.pdf}}




\end{document}
